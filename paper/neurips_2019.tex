\documentclass{article}

\usepackage{bm}
\usepackage{graphicx, caption, subcaption}
\usepackage{amsmath}
\usepackage{amsfonts}
\usepackage{amssymb}
\usepackage{hyperref}
\usepackage{algorithm2e}
\newtheorem{lemma}{Lemma}
\newtheorem{proof}{Proof}
\newtheorem{theorem}{Theorem}

\usepackage{chngcntr}
\usepackage{apptools}

\AtAppendix{\counterwithin{lemma}{section}}
\AtAppendix{\counterwithin{theorem}{section}}
\AtAppendix{\counterwithin{proof}{section}}
% if you need to pass options to natbib, use, e.g.:
\PassOptionsToPackage{numbers, compress}{natbib}
% before loading neurips_2019

% ready for submission
% \usepackage{neurips_2019}

% to compile a preprint version, e.g., for submission to arXiv, add add the
% [preprint] option:
\usepackage[preprint]{neurips_2019}

% to compile a camera-ready version, add the [final] option, e.g.:
%     \usepackage[final]{neurips_2019}

% to avoid loading the natbib package, add option nonatbib:
%     \usepackage[nonatbib]{neurips_2019}

\usepackage[utf8]{inputenc} % allow utf-8 input
\usepackage[T1]{fontenc}    % use 8-bit T1 fonts
\usepackage{hyperref}       % hyperlinks
\usepackage{url}            % simple URL typesetting
\usepackage{booktabs}       % professional-quality tables
\usepackage{amsfonts}       % blackboard math symbols
\usepackage{nicefrac}       % compact symbols for 1/2, etc.
\usepackage{microtype}      % microtypography

\title{Regularization for Sparse Latent Gaussian Process}

% The \author macro works with any number of authors. There are two commands
% used to separate the names and addresses of multiple authors: \And and \AND.
%
% Using \And between authors leaves it to LaTeX to determine where to break the
% lines. Using \AND forces a line break at that point. So, if LaTeX puts 3 of 4
% authors names on the first line, and the last on the second line, try using
% \AND instead of \And before the third author name.

\author{%
	Rui Meng\thanks{Use footnote for providing further information
		about author (webpage, alternative address)---\emph{not} for acknowledging
		funding agencies.} \\
	Department of Statistics\\
	University of California\\
	Santa Cruz, CA 95060 \\
	\texttt{rmeng1@ucsc.edu} \\
	% examples of more authors
	\And
	Herbert Lee \\
	Department of Statistics\\
	University of California\\
	Santa Cruz, CA 95060 \\
	\texttt{herbie@ucsc.edu} \\
    \And
	Braden Soper \\
	Lawrence Livermore National Laboratory \\
	Livermore, CA \\
	\texttt{soper3@llnl.gov} \\
}

\begin{document}
	
	\maketitle
	
\begin{abstract}
The Gaussian Process Latent Variable Model (GPLVM) is a flexible unsupervised bayesian nonparametric modelling approach which has been applied to many learning tasks such as Facial Expression Recognition, Image Reconstruction, Human pose estimation. Due to poor scaling properties of exact inference methods on GPLVMs, approximation methods based on sparse Gaussian processes (SGP) and stochastic variational inference (SVI) are necessary for inference on large data sets. One problem in SGP, especially in latent variable models, is that the distribution of inducing inputs may exhibit overdispersion which may lead to inefficient inference and poor model fit. In this paper, we first propose a regularization approach for latent sparse Gaussian processes in SVI, which balance the distribution of inducing inputs and latent inputs. We justify the use of this regularization term by proving that performing variational inference (VI) on a GPLVM with this regularization term is equivalent to perform VI on a related empirical Bayes model with a prior on inducing inputs. Second, we extend the categorical latent Gaussian process to model categorical time series and illustrate that our regularization is particularly important in this latent dynamic model with a synthetic data set. Finally, we apply our model to a real data set of stock indices and visualize the latent dynamics.
\end{abstract}

\section{Introduction}
Gaussian processes (GP) is generalization of a multivariate Gaussian distribution an can be seen as a stochastic random process on general continuous function. Due to its flexibility, it is widely applied into various fields such as geostatistics \citep{Cressie_1993}, multitask-learning \citep{Banerjee_2008} and reinforcement learning \citep{Rasmussen_2004}. Gaussian process regression and classification are deeply studied in \cite{Rasmussen_2005}.

Although GP is flexible, its exact inference is expensive with the time complexity $O(n^3)$, where $n$ is the number of data points. This renders GP inference infeasible for large real-world datasets. Numerous approximations based on inducing points, named sparse Gaussian process (SGP) methods, have been proposed to avoid the computational issue. Predictive process (PP/DTC) is proposed to approximate GP throughout introducing inducing variables by Seeger \citep{Seeger_2003}. It reduces the time complexity from $O(n^3)$ to $O(nm^2)$ where $m$ is the number of inducing variables. \cite{Snelson_2006} propose fully independent training conditional (FITC) approximation as one of most efficient approximation methods. It corresponding Bayesian approach is proposed as modified predictive process (MPP), which corrects the bias brought from the PP in \cite{Finley_2009}. Moreover, \cite{Snelson_2007} propose partially independent training conditional (PITC) approximation and \cite{Csato_2002} propose expectation propagation pseudo-point approximation. In most approximation approaches, the location of inducing points are optimized based on gradient-based optimization. From Bayesian perspective, \cite{Raj_2011} discuss the inducing input selection using MCMC sampling. On the other hand, \cite{Titsias_2009} applied variational inference to SGP, where it marginalize the optimal variational distribution of inducing variables. While, \cite{Hensman_2012, Hensman_2013} directly optimize variation distribution of inducing variables and latent inputs to gain computational benefits.

The Gaussian process latent variable model (GPLVM) \citep{Lawrence_2003} is proposed by Lawrence as a probabilistic dimensionality reduction method. This method extends the linear mappings from embedding space in dual probabilistic principle component analysis (DPPCA) to nonlinear mappings \citep{Lawrence_2003, Lawrence_2005}. \cite{Lawrence_2005} also discuss its relationship with other dimensionality reduction methods such as Multidimensional Scaling \citep{Mardia_1979} and Kernel PCA \citep{Scholkopf_1998}. Due to the poor scaling property, \cite{Titsias_2010} propose Bayesian GPLVM using variational inference on SGP in \cite{Titsias_2009}. And \cite{Hensman_2013} propose Stochastic variational inference on latent SGP. Many variants of GPLVM are studied in \cite{Lawrence_2007_HGP, Lawrence_2006, Urtasun_2007}.

The main contribution of this work is to propose a regularization approach for latent SGP \citep{Hensman_2013}, which balance the distribution of inducing inputs and latent inputs and contribute to better model prediction. Theoretically, we justify that the use of this regularization term by proving that performing variational inference (VI) on a GPLVM with this regularization term is equivalent to perform VI on a related empirical Bayes model with a prior on inducing inputs. Moreover, we extend categorical latent Gaussian process \citep{Gal_2015} to modeling categorical time series by incorporating dynamical prior \citep{Damianou_2016}. We illustrate that our regularization is particularly import in this dynamic model.

The rest of this paper is organized as follows. In Section \ref{sec:SGP} we show the distribution of inducing points in SGP is important for model prediction. We propose regularization approach for latent SGP and justify the its relation with a related empirical Bayesian model in Section \ref{sec:R}. In Section \ref{sec:TCLGP}, we define the temporal categorical latent Gaussian process (TCLGP) including its variational lower bound, predictive density and algorithm for inference of latent variables. Section \ref{sec:E} applies TCLGP with regularization on both a synthetic dataset and a real data set of stock indices and demonstrate the ability and necessity of regularization in latent SGP. Finally, we summarize our work and discuss its implications in Section \ref{sec:C}.


\section{Sparse Gaussian Process} \label{sec:SGP}
In this section, we show the importance of indicing inputs in SGP. From variational inference perspective, there are two efficient approaches named SGPR and SVGP. SGPR marginalizes the optimal variational distribution of inducing variables \citep{Titsias_2009} while SVGP directly models and optimizes the variational distribution of inducing variables \citep{Hensman_2013}. Furthermore, assume we have $N$ observations and $M$ inducing points , the computation complexity of lower bound in SGPR is $O(M^2N)$ while that in SVGP is $O(M^3)$. Generally, the inducing inputs are optimized by maximizing the corresponding variational bound. However, when inducing inputs are intractable in optimization, the distribution of inducing inputs should capture the distribution of co-variate inputs for better model prediction \cite{Raj_2011}. We illustrate it on a 1-D synthetic data, where we uniformly generate $100$ inputs $\bm x$ on unit interval $(0,1)$. Then the corresponding observations are generated from 
\begin{eqnarray}
y & \sim & \mathcal{N}(y|f, 0.1^2) \nonumber \\
f & = & \sin(2x) + 0.2\cos(22x) \nonumber \,.
\end{eqnarray} 
We take $100$ even-spaced inputs on $(0,1)$ as test inputs and generate corresponding outputs as their true test outputs. Moreover, we use linear combination of Matern kernel and linear kernel as covariance function and take different settings of inducing points, in which inducing points are evenly distributed on a $C-$length interval, centralized at $0.5$, $C = 0.5, 1, 2, 5$. We fix those inducing points in optimization then the predictive posterior processes are shown in Figure\ref{fig:SGP}. And the likelihood and root mean square error are summarized in Table~\ref{tab:SGP}. It illustrates that as the distribution of inducing inputs capture the distribution of inputs, the model has best predictive performance.
\begin{figure}[ht!]
	\centering
	\includegraphics[width = 0.24\linewidth]{pic/1D_sim_SVGP_Fixed_Z0}
	\includegraphics[width = 0.24\linewidth]{pic/1D_sim_SVGP_Fixed_Z1}
	\includegraphics[width = 0.24\linewidth]{pic/1D_sim_SVGP_Fixed_Z2}
	\includegraphics[width = 0.24\linewidth]{pic/1D_sim_SVGP_Fixed_Z5}
	\caption{Stochastic variational Gaussian process on 1D synthetic data with different setting inducing inputs.}
	\label{fig:SGP}
\end{figure} 
\begin{table}
	\centering
	\begin{tabular}{|c|c|c|c|c|}
		\hline
		Length & 0.5 & 1 & 2 & 5 \\
		\hline
		$\ell$ & 0.5226 & 1.7514 & 0.5400 & 0.4976 \\
		\hline 
		RMSE & 0.1424 & 0.0383 & 0.1406 & 0.1463 \\
		\hline
	\end{tabular}
    \caption{Log likelihood ($\ell$) and root mean square error (RMSE) for model prediction on 1-D synthetic data with different setting of fixed inducing points.}
    \label{tab:SGP}.
\end{table}
	
\section{Regularization for Latent Sparse Gaussian Process}	\label{sec:R}
Gaussian process latent variable model is a powerful dimensionality reduction approach. Due to the poor scaling property, sparse Gaussian process is introduced in \cite{Titsias_2010, Hensman_2013}. 

Suppose $Y \in \mathbb{R}^{N\times D}$ be the observed data with latent variables $F \in \mathbb{R}^{N\times D}$, where $N$ is the number of observations and $D$ the dimensionality of observations. And observations have corresponding latent variables $X \in \mathbb{R}^{N \times Q}$ where $Q$ is the dimensionality on latent space. Assume the independence across features, the GPLVM is 
\begin{eqnarray}
y_{nd}|f_{nd} & \sim & \mathcal{N}(y_{nd}|f_{nd}, \sigma^2 = \beta^{-1}) \nonumber \\
f_{nd} & = & \mathcal{F}_d(\bm x_n) \nonumber \\
\mathcal{F}_d & \stackrel{iid}{\sim} & \mathrm{GP}(0, C(\bm\theta)) 
\label{GPLVM}
\end{eqnarray}
with normal prior of latent variables $X$, $
p(X) = \prod_{n = 1}^{N}\mathcal{N}(\bm x_n|\bm 0, I_Q)$
where $\bm x_n$ is the $n$th row of $X$. \cite{Titsias_2009, Titsias_2010} apply variational sparse GP formulation by introducing $D$ separate sets of $M$ inducing variables $U \in \mathbb{R}^{M\times D}$ evaluated at a set of inducing inputs $Z\in\mathbb{R}^{M\times Q}$. Then \cite{Titsias_2010, Hensman_2013} propose the same variational structure 
\begin{eqnarray}
q(F, U, X) & = & \prod_{d = 1}^D\left(p(\bm f_d| \bm u_d, X) q(\bm u_d)\right) q(X)\,,
\label{GPLVM_VD}
\end{eqnarray}
where $\bm f_d$ is the $d$th column of $F$ and $\bm u_d$ is the $d$th column of $U$. Specifically, $X$ have variational distribution $q(X) = \prod_{n = 1}^N\mathcal{N}(\bm x_n|\bm \mu_n, \Sigma_n)$ and $U$ have variational distribution $q(U) = \prod_{d = 1}^D\mathcal{N}(\bm u_d| \bm m_d, S_d)$. 
Then the evidence lower bound (ELBO) is 
\begin{eqnarray}
\mathrm{ELBO} = \sum_{d = 1}^{D}E_{q(F, U, X)}p(\bm y_d|\bm f_d) - \mathrm{KL}(q(U)||p(U)) - \mathrm{KL}(q(X)||p(X)) 
\end{eqnarray}	
\cite{Titsias_2010} derive the variational bound by marginalizing the optimal $q(U)$ based on SGPR in \cite{Titsias_2009}, while \cite{Hensman_2013} derive that by maximizing both parameterized $q(U)$ and $q(X)$ based on SVGP. In the reminder of this paper, we discuss the regularization based on SVGP, due to its computational merit. 

However, for some complicated large datasets, stochastic variational inference cannot help inducing inputs capture the distribution of latent inputs. Take Anuran Calls data set as example, after optimization, the latent inputs are displayed in Figure~\ref{fig:Anuran}. More details are shown in Section~\ref{sec:E}. 

Given this concern, we propose a regularization approach in Section~\ref{sec:r} and build relation between this regularization and Bayesian theory in Section~\ref{sec:rt}.

\subsection{Regularization} \label{sec:r}
In order to take inducing inputs capture the distribution of latent inputs, it is necessary to propose a measurement for the distance between the distribution of inducing inputs and the distribution of latent inputs and punish the distance in the objective function. Generally, the modified lower bound is defined as
\begin{eqnarray}
\mathrm{MELBO} = \mathrm{ELBO} - \lambda R
\label{MELBO}
\end{eqnarray} 
where $\lambda$ is a regularization weight and $R$ is a regularization term which measure the distance between the distribution of latent inputs $X$ and distribution of inducing inputs $Z$. As $\lambda$ increase, the optimization emphasize more on the balance of distributions of inducing inputs and latent inputs.

Specifically, we build a global model for the variational mean of $X$ such that every $\bm \mu_n$ have independent identical Gaussian distribution $p_X(\bm \mu_n) = \mathcal{N}(\bm \mu_n| \bm \mu_{\bm \mu}, \Sigma_{\bm \mu})$ and another global model for inducing points $Z$ such that every $\bm z_m$ have independent identical distribution $p_Z(\bm z_m) = \mathcal{N}(\bm z_m| \bm \mu_Z, \Sigma_Z)$. Then given $\bm \mu$ and $Z$, we derive the maximum likelihood estimates $\hat{\bm\mu}_{\bm \mu}, \hat{\Sigma}_{\bm \mu}$, $\hat{\bm \mu}_Z$ and $\hat{\Sigma}_Z$ using mean and covariance matrix of $\{\bm \mu_n\}$ and $\{\bm z_m\}$. Therefore, the we derive $q_X = \mathcal{N}(\hat{\bm\mu}_{\bm\mu}, \hat{\bm\Sigma}_{\bm \mu})$ to summarize global distribution of latent inputs and derive
$q_Z = \mathcal{N}(\hat{\bm\mu}_Z, \hat{\bm\Sigma}_Z)$ to summarize global distribution of inducing inputs $Z$.

We define the regularization term $R$ by the Kullback-Leibler Divergence distance between $q_X$ and $q_Z$. Mathematically, we define it as
\begin{eqnarray}
R = \mathrm{KL}(q_Z||q_X)\,.
\end{eqnarray}

In \ref{MELBO}, $\lambda$ can be chosen by cross validation or be set as the number of inducing points as a thumb rule. As $\lambda = M$, we justify that performing VI on the GPLVM with regularization is equivalent to perform VI on a related empirical Bayes model with a prior on inducing inputs in Section~\ref{sec:rt}.

\subsection{Regularization Theory} \label{sec:rt}
This section discusses the underlying relation between regularization in GPLVM and a related empirical Bayesian model. First, we display a related empirical Bayesian model with a prior on inducing inputs $Z$ and derive its variational lower bound . Then we illustrate maximizing the MELBO is equivalent to maximizing the variational lower bound in the empirical Bayesian model.

The related empirical Bayesian model is extended from (\ref{GPLVM}) and (\ref{GPLVM_VD}). We put an informative prior on inducing inputs and propose a variational distribution on them as 
\begin{eqnarray}
\bm z_m & \sim & \mathcal{N}(\bm z_m| \hat{\bm \mu}_{\bm\mu}, \hat{\bm\Sigma}_{\bm\mu}) \nonumber \\
q(\bm z_m) & = & \mathcal{N}(\bm z_m|\bm \nu_m, \Upsilon_m) \nonumber
\end{eqnarray}
where $\hat{\bm \mu}_X$, $\hat{\Sigma}_X$ are mean and covariance matrix of $\bm \mu$. The variation joint distribution is defined as $q(F, U, X, Z) = q(Z)q(X)q(U)p(F|Z, X, U)$.Then variational lower bound is derived as
\begin{eqnarray}
\log p(Y) & \geq & \mathrm{E}_{q(F, U, X, Z)}\log p(Y|F) - \mathrm{KL}(q(Z)||p(Z)) - \mathrm{KL}(q(X)||p(X)) - \mathrm{KL}(q(U)||p(U)) \nonumber
\end{eqnarray}

We define $\hat{\bm \mu}_{\bm\nu}$ and $\hat{\bm\Sigma}_{\bm\nu}$ as the mean and covariance matrix of $\{\bm \nu_m\}$ and define a distribution family for $q(Z)$ such that $\Upsilon_{m} = \epsilon I$ for $m = 1,\ldots, M$. We assume covariance of $\{\bm \nu_m\}$ is finite, which means $|\hat{\Sigma}_{\bm \nu}| < K$. Then we have following three lemmas and one theorem:

\begin{lemma}
	When $q(\bm z_m) = \mathcal{N}(\bm \nu_m, \epsilon I)$, as $\epsilon \rightarrow 0$, $\bm z_m \stackrel{d}{\rightarrow} \nu_m$.
\end{lemma}

\begin{lemma}
	The variational lower bound in the empirical Bayesian model is derived as 
	\begin{eqnarray}
	\log p(Y) & \geq & E_{q(F, U, X, Z)}\log p(Y|F) - \mathrm{KL}(q(X)||p(X)) - \mathrm{KL}(q(U)||p(U)) - A + B + C \nonumber
	\end{eqnarray}
	where $A = \frac{M}{2}(\log |\hat{\Sigma}_{\bm \mu}| + \log|\hat{\bm\Sigma}_{\bm\nu}| + Q) + \frac{1}{2}\left(\sum_{m = 1}^M(\bm\nu_m - \hat{\bm \mu}_{\bm \mu})^T\hat{\Sigma}_{\bm\mu}^{-1}(\bm\nu_m - \hat{\bm \mu}_{\bm \mu})\right)$, $B = \frac{M}{2}(Q\log\epsilon-\log K)$ and $C = \frac{2\epsilon}{M\mathrm{tr}(\hat{\Sigma}_{\bm\mu}^{-1})}$.
\end{lemma}
	
\begin{lemma}
	We derive the regularization term as  $M\mathrm{KL}(q_Z || q_X) = \frac{M}{2}(\log |\hat{\Sigma}_{\bm \mu}| + \log|\hat{\bm\Sigma}_{\bm z}| + Q) + \frac{1}{2}\left(\sum_{m = 1}^M(\bm z_m - \hat{\bm \mu}_{\bm \mu})^T\hat{\Sigma}_{\bm\mu}^{-1}(\bm z_m - \hat{\bm \mu}_{\bm \mu})\right)$.
\end{lemma}

\begin{theorem}
	As $\epsilon \rightarrow 0$, maximizing the variational lower bound in empirical Bayesian model is equivalent to maximizing the MELBO in the GPLVM with respect to $Z, q(X)$ and $q(U)$.
\end{theorem}

\section{Temporal Categorical Latent Gaussian Process} \label{sec:TCLGP}
\subsection{Model}
The temporal categorical latent Gaussian process (TCLGP) is extended from categorical latent Gaussian process in \cite{Gal_2015}. We incorporate dynamic priors \citep{Lawrence_2007_HGP, Damianou_2016} to model categorical time series. Assume we have observations $Y \in \mathbb{Z}^{N\times D \times T}$ with time stamp $C \in \mathbb{R}^{N\times T}$. $N$ is the number of individuals, $D$ is feature size and $T$ is the number of time stamps. We assume each feature is categorical with $K$ levels, then our model is expressed as:
\begin{eqnarray}
y_{ndt} & \sim & \mathrm{Cat}(\mathrm{Softmax}(\bm f_{ndt})) \,, \nonumber \\
f_{ndtk} & = & \mathcal{F}_{dk}(\bm x_{nt}) \,, \qquad u_{mdk} = \mathcal{F}_{dk}(\bm z_{m}) \,, \nonumber \\
\mathcal{F}_{dk}(\cdot) & \stackrel{iid}{\sim} & \mathrm{GP}(0,C(\bm \theta_d)), \qquad \bm x_{ntq} = \bm \upsilon_{nq}(C_{nt}) \,, \nonumber \\
\upsilon_{nq}(t) & \stackrel{iid}{\sim} & \mathrm{GP}(C(\bm{\phi}_q)) \nonumber \,,
\end{eqnarray}
where the categorical data have embedding inputs $X \in \mathbb{R}^{N\times T \times Q}$ on a $Q$-dimension latent space. The embedding inputs are latent vectors which summarize all characteristics of corresponding multi-dimensional categorical data.
\subsection{Inference}
The evidence lower bound of the TCLGP is expressed as
\begin{eqnarray}
\log p(Y) & \geq & E_{q(F,X,U)}\log p(Y|F) - \mathrm{KL}(q(X)||p(X|C)) - \mathrm{KL}(q(U)||p(U)) \nonumber 
\end{eqnarray}
where variational distributions of $U$ and $X$ are constructed using independent Gaussian distributions such as 
\begin{eqnarray}
q(U) & = & \prod_{d = 1}^{D}\prod_{k = 1}^{K}\mathcal{N}(\bm u_{dk}| \bm m_{dk}, S_d)\,, \nonumber \\
q(X) & = & \prod_{n = 1}^{N}\prod_{q = 1}^{Q}\prod_{t = 1}^{T} \mathcal{N}(x_{nqt}| \mu_{nqt}, \sigma_{nqt}^2)\,. \nonumber
\end{eqnarray}
Since there is no close form for the expectation \citep{Gal_2015}, we approximate the integration using Monte Carlo integration method. As for a large data set, we propose stochastic variational inference algorithm with batch learning in Appendix~\ref{sec: scia}.

\subsection{Prediction}
This section discusses model prediction. TCLGP model has hyper-parameters $\bm\theta, \bm\phi, Z$ and variational parameters $\bm \mu, \bm \sigma^2, \bm m, \bm S$. After model training, we get all estimates $\hat{\varTheta} = (\hat{\bm \theta}, \hat{\bm \phi}, \hat{\bm Z}, \hat{\bm \mu}, \hat{\bm \sigma}, \hat{\bm m}, \hat{\bm S})$. We estimate latent inputs $X$ and inducing variables $U$ using their corresponding variational mean. Then given a new time stamp $t^*$ in any individual $n$, we can estimate the corresponding latent input given $\hat{X}$ as 
\begin{eqnarray}
p(\bm x_n^*|\hat{X}) = \prod_{q = 1}^{Q}\mathcal{N}(S_0 S_1^{-1}\hat{\bm x}_{n\cdot q}, C(t^*; \hat{\bm \phi}) - S_0 S_1^{-1}S_0^T)
\end{eqnarray}
where $S_0 = C(t^*, \bm c_n; \hat{\bm \phi}_q)$ and $S_1 = C(t^*, \bm c_n; \hat{\bm \phi}_q)C(\bm c_n; \hat{\bm \phi}_q)$. After taking the mean as the estimate of latent inputs $\hat{\bm x}_n^*$, we estimate the corresponding outputs $\bm f_n^*$ by $\hat{\bm f}_n^* = E(\bm f_n^*|\hat{\bm x}_n^*, \hat{U})$. Specifically $\hat{f}^*_{ndk} = \hat{a}_{ndk}^*$, where  $\hat{a}_{ndk}^* = \hat{\bm v}^{*T}_{nd}\hat{\Sigma}_{Zd}^{-1}\hat{\bm m}_{dk}$ and $\hat{\Sigma}_{Zd} = C(\hat{Z}; \hat{\bm \theta}_d)$, $\hat{\bm v}_{nd}^* = C(\hat{Z}, \hat{\bm x}_n^*; \hat{\bm \theta}_d)$. Therefore, the predictive distribution is estimated as $\hat{p}(\bm y_n^* = \bm y|\hat{\varTheta}, t^*) = \prod_{d = 1}^D \mathrm{Softmax}(\hat{\bm f}_n^*)[y_d]$.


\section{Experiments} \label{sec:E}
This section we illustrate our regularization on three datasets. First, we illustrate that regularization is necessary in stochastic variational inference of GPLVM, using Anuran Calls data set. Second, we generate categorical time series data from a simple Markov model and demonstrate regularization is particular important in the TCLGP model, even with simple dimensional data. Finally, we apply the TCLGP with regularization on a real data set of stock indices and visualize the latent dynamics.

\subsection{Anuran}
Anuran Calls data set is available from UCI repository under \url{https://archive.ics.uci.edu/ml/datasets/Anuran+Calls+(MFCCs)}, where we have $7195$ instances and each instances have $22$ attributes. Each instance belongs to one of eight Genus types. We model all instances using GPLVM and inference it with regularization. Specifically, we set latent dimension $Q = 5$ and use $M = 20$ inducing points in the latent SVGP model. We choose independent standard Gaussian distribution as the prior distribution of inducing points and use the PCA approach as initialization for inducing inputs.

Under different regularization weight $\lambda$, the latent means are displayed in Figure~\ref{fig:ANURAN}. After enough iterations, their variational lower bounds are 1.66e+5, 1.64e+5, 1.66e+5, 1.76e+5 for SVI in GPLVM with $\lambda = 0 ,20, 50, 100$, respectively. It illustrates that for complicated data such as the Anuran Calls data set ,regularization contributes to better optimization and better model fitting.

\begin{figure}[ht!]
	\centering
	\includegraphics[width=0.24\linewidth]{pic/VBGPLVM}
	\includegraphics[width=0.24\linewidth]{pic/RVBGPLVM_LAM20}
	\includegraphics[width=0.24\linewidth]{pic/RVBGPLVM_LAM50}
	\includegraphics[width=0.24\linewidth]{pic/RVBGPLVM_LAM100}
	\caption{Latent means of GPLVM for Anuran Call data set. Left panel refers to no regularization and the other three have regularization with different weights $\lambda = 20, 50, 100$. Different colors denote different genus types and crosses denote inducing inputs.}
	\label{fig:ANURAN}
\end{figure}
 
\subsection{Synthetic Time Series}
In this section, We propose a simple Markov model to generate categorical time-series. Suppose categorical data have two dimensions, in which the first dimension has two levels and the second dimension has three levels. The two dimensions are modeled independently. We propose transition matrices in the two dimensions as $\begin{bmatrix}
0.9 & 0.1\\
0.1 & 0.9
\end{bmatrix}$, 
and$ \begin{bmatrix}
0.7, & 0.2 & 0.1 \\
0.1 & 0.8 & 0.1\\
0.1 & 0.2 & 0.7
\end{bmatrix}$ separately. We simulate initial levels uniformly and then generate time series with size $10$ from the simple Markov model and set time stamps evenly distributed over an unit interval. Then we repeat it $100$ times.

For each time series, we take first 9 data for training and 
take the last data for testing. We train our model with and without regularization. We set $M = 20$ inducing points and take latent dimension size $Q = 2$. We initialize the inducing inputs with independent standard normal distributions and initialize the latent input mean with zeros. After training, all latent variables of training data are displayed in Figure~\ref{fig:SIM}. And their ELBO and MELBO values are summarized in Table~\ref{tab:SIM}. It illustrates that inducing inputs have bad coverage on latent inputs without regularization. As $\lambda$ increase, the coverage become better but its evidence lower bound (ELBO) decreases. The ELBO with $\lambda = 100$ is significantly smaller than ELBO with $\lambda = 20$ or $50$. Furthermore, we set $Q = 5$ and repeatedly run $100$ times experiments. The prediction accuracy of GPLVM with and without regularization are $74.34\%(4.13\%)$, $75.79\%(4.34\%)$ and $77.01\%(4.61\%)$ for $\lambda = 0, 20$ and $50$ respectively, where values in the bracket represents standard deviation. It illustrates that regularization contributes to better model prediction.

\begin{figure}[ht!]
	\centering
	\includegraphics[width=0.24\linewidth]{pic/latent_space_SIM}
	\includegraphics[width=0.24\linewidth]{pic/latent_space_SIM_LAM20}
	\includegraphics[width=0.24\linewidth]{pic/latent_space_SIM_LAM50}
	\includegraphics[width=0.24\linewidth]{pic/latent_space_SIM_LAM100}
	\caption{Latent variables of TCLGP for synthetic data. Left panel refers to no regularization and the other three have regularization with different weights $\lambda = 20, 50, 100$.  Different colors denote different observations and crosses denote inducing inputs.}
	\label{fig:SIM}
\end{figure}

\begin{table}[ht!]
	\centering
	\begin{tabular}{|c|c|c|c|c|}
		\hline
		$\lambda$ & 0 & 20 & 50 & 100 \\
		\hline
		ELBO & -1481.39  & -1543.14 & -1566.72 & -2034.46 \\
		\hline
		MELBO &  & -1550.72 & -1579.90 & -2018.77 \\
		\hline
	\end{tabular}
	\caption{Evidence lower bound (ELBO) and modified evidence lower bound of GPLVMs with and without regularization for synthetic categorical time series.}
	\label{tab:SIM}
\end{table}

\subsection{Stock Index}
In this section, we apply the TCLGP model with regularization to a real data set of stock indices.

\subsubsection{Data Description}
Stock indices include SP500, Nikkei225, and DAX stock. Index records are from 6 January 1965 to 5 December 2012. We share the same data in \cite{Joao_2014}, but we take monthly stock indices and treat each year as an independent time series. Letting $\bm y_{nd}$ be monthly stock values for the $n$th year's and the $d$th stock indices, we totally have $48$ independent time series, where each time series contains three-dimensional stock indices for 12 months. Furthermore, we pre-process data by evenly splitting all monthly return rate into five categories, via $20\%, 40\%, 60\%, 80\%$ quantiles, shown in Table~\ref{quan}. Category 1-5 represent bear, slight bear, normal, slight bull and bull market. Each time series randomly select one month as testing data and treat the remaining data as training data.

\begin{table}[ht!]
	\centering
	\begin{tabular}{|c|c|c|c|c|c|c|}
		\hline
		& 0\% & 20\% & 40\% & 60\% & 80\% & 100\% \\
		\hline
		SP500 & -21.9 & -2.5 & -0.1 & 1.8 & 3.7 & 16.5 \\
		\hline
		Nikkei225 & -19.8 & -3.9 & -0.5 & 2.2 & 4.8 & 20.1 \\
		\hline
		DAX & -23.4 & -3.5 & -0.4 & 2.1 & 5 & 21.9 \\
		\hline
	\end{tabular}
    \caption{Percentiles of return rate for three stock indices from 1965 to 2012.}
	\label{quan}
\end{table}

\subsubsection{Hyper-parameters Analysis}
In this dataset, model is sensitive to the hyper-parameters of GP of latent function prior. We fixed them and optimize other model parameters with regularization. We take $M = 20$ and assume same $\bm \phi$ for each dimension $q$. We take different settings of $\bm \phi$ and set $\lambda =20$ using  the rule of thumb. Model fitting is illustrated in MELBO and model prediction is demonstrated using training/testing predictive accuracy and  mean absolute difference in Table~\ref{tab:STOCK}, where predictive accuracy and mean absolute difference for training and testing are defined as 
\begin{eqnarray}
\mathrm{TRPA} & = & \frac{\sum_{n = 1}^{N}\sum_{t = 1}^{T-1}\bm 1(\bm y^{train}_{n\cdot t} = \hat{\bm y}^{train}_{n\cdot t})}{N(T-1)} \,, \nonumber \\
\mathrm{TRMDA} & = & \frac{\sum_{n = 1}^{N}\sum_{d = 1}^{D}\sum_{t=1}^{T-1}|y^{train}_{ndt}-\hat{y}^{train}_{ndt}|}{NDT} \,, \nonumber \\
\mathrm{TPA} & = & \frac{\sum_{n = 1}^{N}\bm 1(\bm y^{test}_{n\cdot} = \hat{\bm y}^{test}_{n\cdot})}{N} \,, \nonumber \\
\mathrm{TMDA} & = & \frac{\sum_{n = 1}^{N}\sum_{d = 1}^{D}|y^{test}_{nd}-\hat{y}^{test}_{nd}|}{ND} \,. \nonumber
\end{eqnarray} 

\begin{table}[ht!]
	\centering
	\begin{tabular}{|c|c|c|c|c|c|c|}
		\hline
		$\phi_{\sigma^2}$ & \multicolumn{3}{|c|}{$0.5$} & \multicolumn{3}{|c|}{$2$} \\
		\hline
		$\phi_l$ & 0.01 & 0.05 & 0.1 & 0.01 & 0.05 & 0.1  \\
		\hline
		MELBO & -2698.79 & \textbf{-2671.42} & -2947.10 & -2691.02 & -2710.76 & -2920.42\\
		\hline
		TRPA & 0.69 & 0.70 & 0.71 & \textbf{0.72} & 0.71 & 0.71  \\
		\hline
		TRMAD & 0.45 & \textbf{0.42} & 0.49 & 0.46 & 0.48 & 0.50 \\
		\hline
		TPA  & 0.21 & \textbf{0.24} & 0.18 & 0.21 & 0.21 & 0.18 \\
		\hline
		TMAD & 1.53 & \textbf{1.45} & 1.65 & 1.53 & 1.53 & 1.65 \\
		\hline
	\end{tabular}
    \caption{Model fitting and prediction results of TCLGP with regularization under different settings of hyper-parameters. We evaluate model fitting by modified evidence lower bound (MELBO) and evaluate model prediction by training predictive accuracy (TRPA), training mean absolute difference (TRMAD), testing predictive accuracy(TRA) and testing mean absolute difference (TMAD).}
	\label{tab:STOCK}
\end{table}

Table~\ref{tab:STOCK} shows as $\phi_{\sigma^2} = 0.5$ and $\phi_l = 0.05$, the model has the best fitting and best performance on prediction except of training predictive accuracy. Therefore, we choose this setting as optimal setting and then continue to Latent process visualization.




\subsubsection{Latent Processes Visualization}
The section is under the optimal model. To visualize the latent process, we denote predictive categories given a certain latent space $\bm x^*$ as $\bm y^* \in [0,\ldots,4]^3$. We let $\tilde{y}^* = \sum \bm y^* \in [0, \ldots, 12]$ to represent market status. The larger value represents that it is more likely to be a bull market. Then we plot a contour using $(\bm x^*, \tilde{y}^*)$ on the latent space shown in Figure~\ref{fig:stock}. On the surface, light colors indicate a bull market while dark colors indicate a bear market. Furthermore, we display predictive posterior latent processes as well as estimated latent traces for both Year 2008 and Year 2009 in Figure~\ref{fig:stock}. Estimated latent trace show that it always stays in dark areas in 2008 while it always stays in light areas. The result exactly matches the fact that a financial crisis happened in 2008 leading the US stock market to a bear market while the US economy returned to normal in 2009. And predictive sensitivity for all years is captured by the predictive posterior processes in the middle two columns in Figure~\ref{fig:stock}.

\begin{figure}[ht!]
	\centering
	\begin{subfigure}[b]{0.24\textwidth}
		\includegraphics[width=\textwidth]{pic/2008year_cat}
	\end{subfigure}
	\begin{subfigure}[b]{0.24\textwidth}
		\includegraphics[width=\textwidth]{pic/year2008_D0}
	\end{subfigure}
	\begin{subfigure}[b]{0.24\textwidth}
		\includegraphics[width=\textwidth]{pic/year2008_D1}
	\end{subfigure}
	\begin{subfigure}[b]{0.24\textwidth}
		\includegraphics[width=\textwidth]{pic/year2008_trace}
	\end{subfigure}
	\begin{subfigure}[b]{0.24\textwidth}
		\includegraphics[width=\textwidth]{pic/2009year_cat}
	\end{subfigure}
	\begin{subfigure}[b]{0.24\textwidth}
		\includegraphics[width=\textwidth]{pic/year2009_D0}
	\end{subfigure}
	\begin{subfigure}[b]{0.24\textwidth}
		\includegraphics[width=\textwidth]{pic/year2009_D1}
	\end{subfigure}
	\begin{subfigure}[b]{0.24\textwidth}
		\includegraphics[width=\textwidth]{pic/year2009_trace}
	\end{subfigure}
	\caption{Categorical plot (left) and predictive posterior latent process (middle) and estimated latent tracing (right) of stock indices in Year 2008 and Year 2009.}
	\label{fig:stock}
\end{figure}

\section{Conclusion} \label{sec:C}
Regularization in GPLVM is necessary when dealing with complicated high dimensional data. It contributes to global optimization in model fitting. Also, we justify the use of regularization by proving that performing VI on a GPLVM with this regularization is equivalent to perform VI on a related empirical Bayes model with a prior on inducing inputs. Without cross validation for regularization weight $\lambda$, we give a rule of thumb for the selection of regularization by setting $\lambda = M$. We propose temporal categorical latent Gaussian process model and illustrate that our regularization is particular important in this latent dynamic model. Finally, we illustrate the its interesting application in the real data set of stock indices.



















































\bibliographystyle{apalike}
\bibliography{ref}

\newpage
\appendix
\appendix

\section{Regularization Theorems} 
\label{sec: rt}
\begin{lemma}
	When $q(\bm z_m) = \mathcal{N}(\bm \nu_m, \epsilon I)$, as $\epsilon \rightarrow 0$, $\bm z_m \stackrel{p}{\rightarrow} \bm \nu_m$.
\end{lemma}

\begin{proof} Since
	\begin{eqnarray}
	\lim\limits_{\epsilon \rightarrow 0} p(|\bm z_m - \bm \nu_m| > \epsilon_0) & = & \lim\limits_{\epsilon \rightarrow 0} p(|\frac{\bm z_m - \bm \nu_m}{\epsilon}| > \frac{\epsilon_0}{\epsilon}) \nonumber \\
	& = & 2\lim\limits_{\epsilon \rightarrow 0} (1 - \Phi(\frac{\epsilon_0}{\epsilon}))^{Q} \nonumber \\
	& = & 0 \,, \nonumber
	\end{eqnarray}
	we conclude that $\bm z_m \stackrel{p}{\rightarrow} \bm \nu_m$.
\end{proof}

\begin{lemma}
	The variational lower bound in the empirical Bayesian model is derived as 
	\begin{eqnarray}
	\log p(Y) & \geq & E_{q(F, U, X, Z)}\log p(Y|F) - \mathrm{KL}(q(X)||p(X)) - \mathrm{KL}(q(U)||p(U)) - A + B + C \nonumber
	\end{eqnarray}
	where $A = \frac{M}{2}(\log |\hat{\Sigma}_{\bm \mu}| + \log|\hat{\bm\Sigma}_{\bm\nu}| + Q) + \frac{1}{2}\left(\sum_{m = 1}^M(\bm\nu_m - \hat{\bm \mu}_{\bm \mu})^T\hat{\Sigma}_{\bm\mu}^{-1}(\bm\nu_m - \hat{\bm \mu}_{\bm \mu})\right)$, $B = \frac{M}{2}(Q\log\epsilon-\log K)$ and $C = \frac{2\epsilon}{M\mathrm{tr}(\hat{\Sigma}_{\bm\mu}^{-1})}$.
\end{lemma}

\begin{proof}
	{\footnotesize
		\begin{eqnarray}
		\log p(\bm Y) & \geq &  E_{q(F, U, X, Z)}\log p(Y|F) - \mathrm{KL}(q(\bm X)|| p(\bm X)) - \mathrm{KL}(q(\bm U) ||p(\bm U )) -\mathrm{KL}(q(\bm Z)|| p(\bm Z)) \nonumber \\
		& = & E_{q(F, U, X, Z)}\log p(Y|F) - \mathrm{KL}(q(\bm X)|| p(\bm X)) - \mathrm{KL}(q(\bm U) ||p(\bm U )) - A \nonumber \\
		& & + \frac{M}{2}(Q\log\epsilon-\log |\hat{\bm\Sigma}_{\bm \nu}|) + C \nonumber \\
		& \geq & E_{q(F, U, X, Z)}\log p(Y|F) - \mathrm{KL}(q(X)||p(X)) - \mathrm{KL}(q(U)||p(U)) - A + B + C \nonumber
		\end{eqnarray}
	}
\end{proof}

\begin{lemma}
	We derive the regularization term as $M\mathrm{KL}(q_Z || q_X) = \frac{M}{2}(\log |\hat{\Sigma}_{\bm \mu}| + \log|\hat{\bm\Sigma}_{\bm z}| + Q) + \frac{1}{2}\left(\sum_{m = 1}^M(\bm z_m - \hat{\bm \mu}_{\bm \mu})^T\hat{\Sigma}_{\bm\mu}^{-1}(\bm z_m - \hat{\bm \mu}_{\bm \mu})\right)$.
\end{lemma}

\begin{proof}
	{\footnotesize
		\begin{eqnarray}
		\mathrm{KL}(q_Z || q_X) & = & \frac{1}{2}\left[\log\frac{|\hat{\Sigma}_{\bm\mu}|}{|\hat{\Sigma}_Z|} - Q + \mathrm{tr}(\hat{\Sigma}_{\bm\mu}^{-1}\hat{\Sigma}_Z)+(\hat{\bm \mu}_{\bm\mu} - \hat{\bm \mu}_Z)^T\hat{\Sigma}_{\bm\mu}^{-1}(\hat{\bm \mu}_{\bm\mu} - \hat{\bm \mu}_Z)\right] \nonumber \\
		& = & \frac{1}{2}\left[\log\frac{|\hat{\Sigma}_{\bm\mu}|}{|\hat{\Sigma}_Z|} - Q + \mathrm{tr}\left(\hat{\Sigma}_{\bm\mu}^{-1}((\hat{\bm \mu}_{\bm\mu} - \hat{\bm \mu}_Z)(\hat{\bm \mu}_{\bm\mu} - \hat{\bm \mu}_Z)^T + \hat{\Sigma}_Z)\right)\right] \nonumber \\
		& = & \frac{1}{2}\Bigg[\log\frac{|\hat{\Sigma}_{\bm\mu}|}{|\hat{\Sigma}_Z|} - Q + \frac{1}{M}\mathrm{tr}\bigg(\hat{\Sigma}_{\bm\mu}^{-1}(M(\hat{\bm \mu}_{\bm\mu} - \hat{\bm \mu}_Z)(\hat{\bm \mu}_{\bm\mu} - \hat{\bm \mu}_Z)^T + \sum_{m = 1}^{M}(\bm u_m - \hat{\bm \mu}_Z)(\bm u_m - \hat{\bm \mu}_Z)^T\bigg)\Bigg] \nonumber \\
		& = & \frac{1}{2}\Bigg[\log\frac{|\hat{\Sigma}_{\bm\mu}|}{|\hat{\Sigma}_Z|} - Q + \frac{1}{M}\mathrm{tr}\bigg(\hat{\Sigma}_{\bm\mu}^{-1}(M\hat{\bm \mu}_{\bm\mu}\hat{\bm \mu}_{\bm\mu}^T - M\hat{\bm \mu}_{\bm\mu}\hat{\bm \mu}_Z^T - M\hat{\bm \mu}_Z\hat{\bm \mu}_{\bm\mu}^T + M\hat{\bm \mu}_Z\hat{\bm \mu}_Z^T \nonumber \\ 
		& & + \sum_{m = 1}^{M}\bm u_m \bm u_m^T - (\sum_{m = 1}^{M}\bm z_m)\hat{\bm \mu}_Z^T - \hat{\bm \mu}_Z(\sum_{m = 1}^{M}\bm z_m)^T + M\hat{\bm \mu}_Z\hat{\bm \mu}_Z^T \bigg)\Bigg] \nonumber \\
		& = &  \frac{1}{2}\left[\log\frac{|\hat{\Sigma}_{\bm\mu}|}{|\hat{\Sigma}_Z|} - Q + \frac{1}{M}\mathrm{tr}\bigg(\hat{\Sigma}_{\bm\mu}^{-1}(\sum_{m = 1}^{M}\bm z_m\bm z_m^T - M\hat{\bm \mu}_{\bm\mu}\hat{\bm \mu}_Z^T - M\hat{\bm \mu}_Z\hat{\bm \mu}_{\bm\mu}^T + M\hat{\bm \mu}_X\hat{\bm \mu}_{\bm\mu}^T)\bigg)\right]\nonumber \\
		& = &  \frac{1}{2}\Bigg[\log\frac{|\hat{\Sigma}_{\bm\mu}|}{|\hat{\Sigma}_Z|} - Q + \frac{1}{M}\mathrm{tr}\bigg(\hat{\Sigma}_{\bm\mu}^{-1}(\sum_{m = 1}^{M}\bm z_m\bm z_m^T - \hat{\bm \mu}_{\bm\mu}(\sum_{m = 1}^{M}\bm z_m)^T - (\sum_{m = 1}^{M}\bm z_m)\hat{\bm \mu}_{\bm\mu}^T + M\hat{\bm \mu}_{\bm\mu}\hat{\bm \mu}_{\bm\mu}^T)\bigg)\Bigg]\nonumber \\
		& = & \frac{1}{2}\Bigg[\log\frac{|\hat{\Sigma}_{\bm\mu}|}{|\hat{\Sigma}_Z|} - Q + \frac{1}{M}\mathrm{tr}\bigg(\hat{\Sigma}_{\bm\mu}^{-1}(\sum_{m = 1}^{M}\bm z_m\bm z_m^T - \hat{\bm \mu}_{\bm\mu}(\sum_{m = 1}^{M}\bm u_m)^T - (\sum_{m = 1}^{M}\bm z_m)\hat{\bm \mu}_{\bm\mu}^T + M\hat{\bm \mu}_{\bm\mu}\hat{\bm \mu}_{\bm\mu}^T)\bigg)\Bigg]\nonumber \\
		& = & \frac{1}{2}\Bigg[\log\frac{|\hat{\Sigma}_{\bm\mu}|}{|\hat{\Sigma}_Z|} - Q + \frac{1}{M}\mathrm{tr}\bigg(\hat{\Sigma}_{\bm\mu}^{-1}(\sum_{m = 1}^{M}(\bm z_m - \hat{\bm \mu}_{\bm\mu})(\bm z_m - \hat{\bm \mu}_{\bm\mu})^T)\bigg)\Bigg]\nonumber \\
		& = & \frac{1}{M}\sum_{m = 1}^{M}\frac{1}{2}\left[\log\frac{|\hat{\Sigma}_{\bm\mu}|}{|\hat{\Sigma}_Z|} - Q + (\bm z_m - \hat{\bm \mu}_{\bm\mu})^T\hat{\Sigma}_{\bm\mu}^{-1}(\bm z_m - \hat{\bm \mu}_{\bm\mu})\right]\nonumber \,.
		\end{eqnarray}
	}
	
	Therefore, $ M\mathrm{KL}(q_Z || q_X) = \frac{1}{2}\sum_{m = 1}^{M}\left[\log\frac{|\hat{\Sigma}_{\bm\mu}|}{|\hat{\Sigma}_Z|} - Q + (\bm z_m - \hat{\bm \mu}_{\bm\mu})^T\hat{\Sigma}_{\bm\mu}^{-1}(\bm z_m - \hat{\bm \mu}_{\bm\mu})\right] =\frac{M}{2}(\log |\hat{\Sigma}_{\bm \mu}| + \log|\hat{\bm\Sigma}_{\bm z}| + Q) + \frac{1}{2}\left(\sum_{m = 1}^M(\bm z_m - \hat{\bm \mu}_{\bm \mu})^T\hat{\Sigma}_{\bm\mu}^{-1}(\bm z_m - \hat{\bm \mu}_{\bm \mu})\right)$. 
\end{proof}

\begin{theorem}
	As $\epsilon \rightarrow 0$, maximizing the variational lower bound in empirical Bayesian model is equivalent to maximizing the MELBO in the GPLVM with respect to $Z, q(X)$ and $q(U)$.
\end{theorem}

\begin{proof}
    In the empirical Bayesian model, variational parameters are $\bm \mu, \bm \Sigma, \bm m, \bm s, \bm \nu$. We denote all parameters as $\varTheta$. We have $\lim\limits_{\epsilon \rightarrow 0}C = 0$ and 
	\begin{eqnarray}
    \lim\limits_{\epsilon \rightarrow 0}E_{q(F, U, X, Z)}\log p(Y|F) & = & E_{q(F, U, X| Z = \bm \nu)}\log p(Y|F)
	\end{eqnarray}
	Then according to Lemma 2, we derive that
	\begin{eqnarray}
	& \arg\max\limits_{\varTheta}\lim\limits_{\epsilon \rightarrow 0} E_{q(F, U, X, Z)}\log p(Y|F) - \mathrm{KL}(q(X)||p(X)) - \mathrm{KL}(q(U)||p(U)) - A + B + C \nonumber \\
	= & 
	\arg\max\limits_{\varTheta}\lim\limits_{\epsilon\rightarrow 0} E_{q(F, U, X, Z)}\log p(Y|F) - \mathrm{KL}(q(X)||p(X)) - \mathrm{KL}(q(U)||p(U)) - A  \nonumber \\  
	= & \arg\max\limits_{\varTheta} E_{q(F, U, X| Z=\bm \nu)}\log p(Y|F) - \mathrm{KL}(q(X)|| p(X)) - \mathrm{KL}(q(U) ||p(U )) - A \nonumber 
	\end{eqnarray}
	When we replace $\bm \nu$ as $Z$, due to Lemma 3, this optimization is equivalent to maximize $\mathrm{ELBO} - M\mathrm{KL}(q_Z||q_X)$ which is the exactly MELBO. Finally due to Lemma 1, the $q(Z)$ in empirical Bayesian model will converges to the same optimized $Z$ in the GPLVM with regularization.
\end{proof}

\section{Stochastic Variational Inference Algorithm}
\label{sec: scia}
This section displays a general stochastic variational inference algorithm for large datasets. In this framework, we set the number of training epochs $N_{train}$ and evenly divide the whole dataset into $N_{batch}$ clusters. Each cluster includes the observations $\bm Y_i$ and their corresponding time stamp data $\tilde{\bm T}_i$ and their corresponding hyper-parameters of embedding inputs, $\bm m_i$ for the mean and $\bm s_i$ for the standard deviation. In the context of the TCLGP, the model parameters include $\bm \theta, \bm Z, \bm \mu, \bm \Sigma$ and the model inputs include both observable data $\bm Y_i, \tilde{\bm T}_i$ and latent hyper-parameters $\bm m_i, \bm s_i$. Suppose the MELBO(g) is rewritten as
\begin{eqnarray}
\mathrm{MELBO(g)} & = & \mathrm{ELBO(g)} - \lambda\mathrm{R} \nonumber \\
& = & gR_0 + R_1 - \lambda\mathrm{R}\,, \nonumber \\
R_0 & = & -\mathrm{KL}(q(\bm X) || p(\bm X | \tilde{\bm T})) - \mathrm{KL}(q(\bm U) || p(\bm U)) \,, \nonumber \\
R_1 & = & \int q(\bm X) q(\bm  U) p(\bm F | \bm X, \bm U) \log\left(\sum_{n = 1}^{N}\sum_{d = 1}^{D}\sum_{t = 1}^{T} p(\bm y_{ndt} | \bm f_{ndt})\right)d\bm X \bm U \bm F \,, \nonumber
\end{eqnarray}
where $R_0$ and $R_1$ are the regularization term and the reconstruction term 
in the ELBO separately and $\mathrm{R}$ is a regularization term related to inducing inputs, and $g\in[0,1])$ is the annealing factor referred in \cite{Bowman_2015}. The annealing increase factor is denoted as $\Delta g$. Then inference algorithm is displayed as follows: 

\begin{algorithm}[H]
	\SetAlgoLined
	Set $g = 0$\;
	\For{$i = 1$ \KwTo $N_{train}$}{
		\For{$j = 1$ \KwTo $N_{batch}$}{
			Assign both observable data $\bm Y_i, \tilde{\bm T}_i$ and latent data $\bm m_i, \bm s_i$ to the TCLGP model\;
			Update model parameters $\bm \theta, \bm Z, \bm \mu, \bm \Sigma$ and latent hyper-parameters $\bm m_i, \bm s_i$ through maximizing the MELBO(g) using a stochastic gradient descend method.\;
			}
		$g = \min(g + \Delta g, 1)$\;
		}
	\caption{Stochastic variational inference algorithm for large datasets.}
\end{algorithm}



%\section{KL divergence of Multivariate Normal distributions} 
%\label{sec: kl_mn}
%Assume the dimension size is $D$ and $p \sim \mathcal{N}(\bm \mu, \Sigma)$ and $q \sim \mathcal{N}(\tilde{\bm\mu}, \tilde{\Sigma})$. Then KL divergence between $p$ and $q$ is expressed as:
%\begin{eqnarray}
%\mathcal{KL}[p||q] & = & \frac{1}{2}\left(\log\frac{|\tilde{\Sigma}|}{|\Sigma|} - D + \mathrm{tr}(\tilde{\Sigma}^{-1}\Sigma) + (\tilde{\bm \mu} - \bm \mu)^T\tilde{\Sigma}^{-1}(\tilde{\bm \mu} - \bm \mu) \right)
%\label{KL_MN}
%\end{eqnarray}

%\section{KL divergence of Inverse Gamma distribution}
%Assume $p \sim \mathcal{IG}(\alpha, \beta)$ and $q \sim \mathcal{IG}(\tilde{\alpha}, \tilde{\beta})$. Then KL divergence between $p$ and $q$ is expressed as \cite{Beckmann_2016}:
%\begin{eqnarray}
%\mathcal{KL}[p||q] & = & \int_{0}^{\infty} p(x)\log\frac{p(x)}{q(x)}dx \\
%& = & (\alpha - \tilde{\alpha})\Psi(\alpha) + \tilde{\beta}(\frac{\alpha}{\beta}) - \alpha + \log\frac{\beta^{\tilde{\alpha}+1}\Gamma(\tilde{\alpha})}{\beta\tilde{\beta}^{\tilde{\alpha}}\Gamma(\alpha)}
%\end{eqnarray}
%
%\section{KL divergence of Dirichlet distribution}
%Assume $p \sim \mathrm{Dir}(\bm c)$ and $q \sim \mathrm{Dir}(\tilde{\bm c})$. Then KL divergence between $p$ and $q$ is expressed as:
%\begin{eqnarray}
%\mathcal{KL}[p||q] & = & \log\Gamma(c_0) - \sum_{i = 1}^{n}\log\Gamma(c_i) - \log\Gamma(\tilde{c}_0) + \sum_{i = 1}^{n}\log\Gamma(\tilde{c}_i) + \sum_{i = 1}^{n}(c_i - \tilde{c}_i)(\Psi(c_i) - \Psi(c_0))
%\end{eqnarray}
%
%\section{KL divergence of Normal distribution}
%Assume $p \sim \mathcal{N}(\mu, \sigma^2)$ and $q \sim \mathcal{N}(\tilde{ \mu}, \tilde{\sigma}^2)$. Then KL divergence between $p$ and $q$ is expressed as:
%\begin{eqnarray}
%\mathcal{KL}[p||q] & = & \log\frac{\tilde{\sigma}}{\sigma} + \frac{\sigma^2 + (\mu- \tilde{\mu})^2}{2\tilde{\sigma}^2} - \frac{1}{2}
%\end{eqnarray}
%
%
%\section{KL divergence between Uniform distribution and Normal distribution}
%Assume $p \sim \mathcal{U}(a, b)$ and $q \sim \mathcal{N}(\mu, \sigma^2)$. Then KL divergence between $p$ and $q$ is expressed as:
%\begin{eqnarray}
%\mathcal{KL}[p||q] & = & \int_{-\infty}^{\infty}p(x)\log(\frac{p(x)}{q(x)})dx \nonumber \\
%& = & -\int_{a}^{b} \log(q(x))dx \nonumber \\
%& = & (b-a)\log(\sqrt{2\pi}\sigma) + \frac{x^3}{6}|_{\frac{a-\mu}{\sigma}}^{\frac{b-\mu}{\sigma}}
%\end{eqnarray}
%
%
%\section{KL divergence of Categorical distribution}	
%Assume $p \sim \mathrm{Cat}(\bm \pi)$ and $q \sim \mathrm{Cat}(\tilde{\bm \pi})$. Then KL divergence between $p$ and $q$ is expressed as:
%\begin{eqnarray}
%\mathcal{KL}[p||q] & = & \sum_{i = 1}^{n}\pi_i\log\frac{\pi_i}{\tilde{\pi}_i}
%\end{eqnarray}

%\section{Advanced MCMC} \label{sec: Advanced MCMC}
%This section, we introduce three advanced MCMC algorithms. First of all, we introduce traditional Metropolis Hasting framework.
%
%Sampling $\bm \theta \sim p(\bm \theta)$ is of interest. The proposal transition distribution is set as $q(\bm \theta', \bm \theta)$. Then the recursive procedures are displayed as
%\begin{itemize}
%	\item Sample new parameters $\bm \theta_{new}\sim q(\bm \theta_{new}| \bm \theta^{(i)})$.
%	\item Compute the accept rate $r = \min\left(\frac{p(\bm{\theta}_{new})q(\bm{\theta}| \bm{\theta}_{new})}{p(\bm{\theta})q(\bm \theta_{new}| \bm{\theta})}, 1\right)$.
%	\item Accept $\bm \theta^{(i+1)} = \bm{\theta}_{new}$ with probability $r$, otherwise, keep previous parameters $\bm \theta^{(i+1)} = \bm{\theta}^{(i)}$.
%\end{itemize}
%
%The three algorithms are based on the basic Metropolis Hasting framework.
%\subsection{Langevin Adapted Metropolis Hasting}
%To utilize the gradient information, the proposal structure is proposed as the form\cite{Besag_1995}
%\begin{eqnarray}
%\bm \theta_{new} = \bm \theta + \frac{\epsilon^2}{2}g(\bm \theta) + \epsilon\mathcal{N}(\bm 0, \bm I)
%\end{eqnarray}
%where $g(\bm{\theta}) = \triangledown\log p(\bm\theta)$. It suggests the proposal kernel is 
%\begin{eqnarray}
%q(\bm \theta_{new}, \bm \theta) = \exp\left[-\frac{1}{2\epsilon^2}\|\bm \theta - \bm \theta_{new} - \frac{\epsilon^2}{2}g(\bm \theta_{new})\|^2\right]
%\end{eqnarray}
%
%\subsection{Hessian Adapted Metropolis Hasting}
%\cite{Qi_2002} propose the Hessian adaptive Metropolis Hasting approach which utilize both gradient and Hessian matrix information in proposal structure
%\begin{eqnarray}
%\bm \theta_{new} = \bm \theta + \epsilon\Sigma(\bm \theta)g(\bm \theta) + \mathcal{N}(\bm 0, \Sigma(\bm\theta))
%\end{eqnarray}
%where $\Sigma(\bm{\theta}) = -\left(\triangledown\triangledown^T\log p(\bm\theta)\right)^{-1}$. It suggests the proposal kernel is 
%\begin{eqnarray}
%q(\bm \theta_{new}, \bm \theta) = \mathcal{N}(\bm{\theta}_{new} - \bm \theta + \epsilon\Sigma(\bm \theta)g(\bm \theta), \Sigma(\bm \theta))
%\end{eqnarray}
%
%
%\subsection{Fisher Adapted Metropolis Hasting}
%One concern of Hessian Adapted Metropolis Hasting algorithm is that the matrix inverse problem. Numerically the Hessian matrix may not be positive definite. The idea of replace Hessian matrix by negative fisher information matrix has been explored. We proposed a simple proposal structure 
%\begin{eqnarray}
%	\bm \theta_{new} = \bm \theta + \mathcal{N}(\bm 0, \Sigma^*)
%\end{eqnarray}
%where $\Sigma^* = -(\triangledown\triangledown^{T}(\bm \theta^*))^{-1}$ is the inverse of the observed fisher information matrix and $\bm \theta^*$ is the maximum likelihood estimates for $\log p(\bm \theta)$. Then it is a symmetric random walk with proposal distribution such that
%\begin{eqnarray}
%q(\bm \theta_{new}, \bm\theta) = \mathcal{N}(\bm{\theta}_{new} - \bm\theta, \Sigma^*)
%\end{eqnarray}
%where $q(\bm \theta_{new}| \bm \theta)$ and $q(\bm \theta| \bm \theta_{new})$ cancel out in the computation of accept rate. The computation of Fisher Adapted Metropolis Hasting is much cheaper than Hessian Adapted Metropolis Hasting since it does not need to compute the proposal transition kernel in each iteration.
%
%\section{Matrix Gradient} \label{sec: Matrix Gradient}
%Let $a_{ij} \in \mathbb{R}, i = 1, \ldots, m, j = 1, \ldots,n$. Then the real matrix $\bm A$ is expressed as 
%\begin{eqnarray}
%\bm A = \begin{pmatrix}
%a_{11} & a_{12} & \ldots & a_{1n} \\
%a_{21} & a_{22} & \ldots & a_{2n} \\
%\vdots & \vdots & & \vdots \\
%a_{m1} & a_{m2} & \ldots & a_{mn}
%\end{pmatrix}\,.
%\end{eqnarray}
%Assume $f$ is a mapping from $\mathbb{R}^{m \times n}$ to $\mathbb{R}$. Then the gradient of $f(\bm A)$ can be expressed as 
%\begin{eqnarray}
%\frac{\partial}{\partial \bm A}f(\bm A) = \bm A' = \begin{pmatrix}
%\frac{\partial}{\partial a_{11}} & \frac{\partial}{\partial a_{12}} & \ldots & \frac{\partial}{\partial a_{1n}} \\
%\frac{\partial}{\partial a_{21}} & \frac{\partial}{\partial a_{22}} & \ldots & \frac{\partial}{\partial a_{2n}} \\
%\vdots & \vdots & & \vdots \\
%\frac{\partial}{\partial a_{m1}} & \frac{\partial}{\partial a_{m2}} & \ldots & \frac{\partial}{\partial a_{mn}}
%\end{pmatrix} \bm A\,.
%\end{eqnarray}

\end{document}